\documentclass[12pt]{article}

\usepackage[polish]{babel}
\usepackage{csquotes}
\usepackage[a4paper,top=2.5cm,bottom=2.5cm,left=3cm,right=2cm,twoside]{geometry}
\usepackage{indentfirst}
% Useful packages
\usepackage{lipsum}
\usepackage{amsmath}
\usepackage{graphicx}
\usepackage{xurl}
\usepackage[colorlinks=true, allcolors=blue]{hyperref}
\usepackage{float}
\usepackage{fontspec}
\usepackage{pdfpages}
\setmainfont{Times New Roman}
\setlength{\parindent}{1.25cm}
\linespread{1.5}
\setlength{\parskip}{6pt}
\usepackage[sorting=none, style=nature]{biblatex}
\addbibresource{bibliography.bib}
\setcounter{biburllcpenalty}{9000}
\setcounter{biburlucpenalty}{9000}
\babelprovide[transforms=oneletter.nobreak]{polish}
\hypersetup{linkcolor=black}
\usepackage[all]{hypcap}
\addto\captionspolish{\renewcommand{\figurename}{Rys.}}
\addto\captionspolish{\renewcommand{\tablename}{Tab.}}
\renewcommand{\tableautorefname}{Tab.} % PS
\renewcommand{\figureautorefname}{Rys.} % PS
\renewcommand{\listfigurename}{Spis rysunków}
\renewcommand{\listtablename}{Spis tabel}

\input{commands}

\begin{document}
\includepdf[pages={1}]{titlepage.pdf}
\newpage
\
\thispagestyle{empty}
\newpage

\newpage
\clearpage
\tableofcontents

\newpage
\addcontentsline{toc}{section}{Wstęp}
\section*{Wstęp}
\textit{W tym miejscu należy umieścić wstęp opisujący dlaczego ten temat jest istotny oraz motywację studenta w jego wyborze. Poniżej przykład jak zacząć nawiązanie do dziedziny, tak aby nie zaczynać pracy od oklepanej frazy „W dzisiejszych czasach …”:}

Nauka programowania od zawsze uznawana była za bardzo trudne zadanie dla nowicjuszy. Podejmowane przez nich próby pisania kodu często kończyły się niepowodzeniem i szybką rezygnacją.

Techniki biometryczne pozwalają na ochronę zasobów na urządzeniach bez potrzeby wprowadzania pinów, haseł oraz skomplikowanych wzorów, które mogą zostać zapomniane lub przechwycone przez osoby postronne. Wystarczy przyłożyć palec, a urządzenie zostaje odblokowane w ułamku sekundy. Jest to rozwiązanie bardzo wygodne dla użytkowników.

Efektywna komunikacja jest czynnikiem koniecznym do zagwarantowania rozwoju, stabilizacji, jak i również bezpieczeństwa niezależnie od płaszczyzny, na której zachodzi. Aby można było mówić o sukcesie procesu informowania, należy szczegółowo przeanalizować domenę w jakiej ma ona zachodzi oraz wybrać właściwą metodę przesyłania danych.

\textit{Jak widać, w każdym przypadku akapit zaczął się od określenia jakiejś pozytywnej lub negatywnej strony związanej z tematem pracy. Pozwala to na łatwe rozwinięcie treści.}

\textit{Jeśli brakuje wam weny, to wstępie możecie poruszyć opis problemów związanych z tematem, zastosowanie danej technologii, badania rynkowe np. o wzrastającej popularności danej technologii.}

\textit{Poniżej przedstawiłem przykładowe zakończenia z motywacją:}

Na rynku dostępnych jest wiele modeli robotów edukacyjnych, jednak większość 
z nich jest bardzo droga lub trudno dostępna dla szkół. Istnieje więc potrzeba stworzenia optymalnego modelu, łatwego w obsłudze, a zarazem będącego w przystępnej cenie. 

Praca ta ma na celu przedstawienie i porównanie różnych technik analizy interfejsu użytkownika tak, aby to, co wydaje się być intuicyjne i proste dla konstruującego stronę programisty, faktycznie takim się stało dla odbiorcy – użytkownika owego produktu.

\textbf{Całość, razem z zakończeniem powinna mieścić się na jednej stronie.}
\newpage

\addcontentsline{toc}{section}{Cel i zakres pracy}
\section*{Cel i zakres pracy}
Celem tej pracy jest zaprojektowanie i implementacja oprogramowania do przetwarzania obrazów opartego na bibliotece OpenCV z funkcjonalnym interfejsem graficznym.

Zakres pracy objął: analizę istniejących rozwiązań na podstawie której stwierdzono, że nie ma aktualnego i gotowego produktu który realizuje dokładnie założenie zastąpienia programistycznego podejścia do obsługi biblioteki OpenCV. Jest wiele roziązań które realizuje część biblioteki lub jej obsługa nie pozwala na przetwarzanie obrazów w niedestrukcyjny sposób bez dodatkowego wysiłku od użytkownika jak używanie warstw czy inkrementalnych zapisów. 

Jako platformę wybrano komputer osobisty z powodu znaczącej mocy obliczeniowej, dostępu do dużej ilości pamięci RAM oraz dostępu do operacji procesorów x86 \cite{x86opencv}. Jako docelowy system operacyjny wybrano Microsoft Windows 11 ponieważ jest to system używany przez autora tej pracy. Do tworzenia aplikacji skorzystano z języka C\# oraz Windows Presentation Framework. 

Pomysł na ten projekt powstał dzięki doświadczeniu autora pracy w używaniu różnych systemów opartych o edytory typu węzłowego jak Adobe Substance Designer, edytor Blueprintów w Unreal Engine oraz edytor shaderów czy geometry nodes w Blenderze. Ten typ interfejsu pozwala na interakcję użytkownika z wizualnym przedstawieniem jakichś metod, funkcji czy bloków logicznych pozwalających na przesyłanie wartości w podobny sposób jak można robić to w kodzie, ale zdecydowanie bardziej przystępne dla osób które nie mają odpowiedniego doświadczenia w programowaniu.  
\newpage

\section{Rys historyczny}
\subsection{Rys historyczny}

Dziedzina tej pracy jaką jest przetwarzanie obrazów ma początki swojej historii wcześniej niż urządzenia które obecnie nazywamy komputerami.

\subsubsection{Wywoływanie zdjęć}
Przetwarzanie obrazów to nie jest technologia, która mogła zacząć istnieć po wynalezieniu komputerów. Używając aparatów korzystających z kliszy filmowej, po ekspozycji należy go poddać procesowi wywoływania \autoref{fig:wywolywanie}. 
Polega on na wyciągnięciu pierwotnego efektu naświetlenia do obrazu, który oddaje scenę uchwyconą przez aparat \cite{doi:https://doi.org/10.1002/14356007.a20_001}. Proces w przypadku zdjęć aparatem na kliszę polega na zanurzaniu jej w odpowiednich związkach chemicznych na określone ilości czasu by otrzymać zamierzony efekt. 

\begin{figure}[H]
    \centering
    \includegraphics{./images/Picture1.jpg}
    \caption{Wywoływanie biało czarnej kliszy \cite{film}}.
    \label{fig:wywolywanie}
\end{figure}


Istnieją wariacje na temat takiego przetwarzania, różni się ono trochę w zależności od technologii kliszy. W niektórych przypadkach należy wywołać pozytyw zamiast negatywu \cite{almanac}. 
Następnie po wywoływaniu można poddać obróbce dalszymi chemikaliami jak np. siarczek sodu dla uzyskania efektu sepii \cite{sepia}. 

\subsubsection{Transmisja obrazów}
Technologia którą znamy dobrze w obecnych czasach powstała w latach 20. XX wieku dzięki pracy John Logie Baird'a \cite{times1926}. 
Zaprezentował on pierwszą na świecie transmisję telewizyjną na żywo.
W 1928 roku \cite{bairdCompany} firma założona przez niego emitowała też pierwsze nagranie przez atlantyk. 
Wszystko to dzięki przetwarzaniu analogowego sygnału video w celu zmniejszenia ilości danych potrzebnych do wyświetlenia obrazu w telewizji \cite{analogVideo}.

\subsubsection{Cyfrowe przetwarzanie obrazów}
Początki nowoczesnego przetwarzania obrazów zostały stworzone w latach 60. XX w. w Bell Laboratories, Jet Propulsion Laboratory, Massachusetts Institute of Technology i University of Maryland \cite{computerProcessing}. 
Początkowymi obszarami zastosowania były obrazy satelitarne, przesyłanie obrazów przez linie telefoniczne, diagnostyka obrazowa, wideofony, rozpoznawanie znaków i ulepszanie fotografii. 

Inicjalnie dużo skupiano się na ulepszeniu jakości obrazu. Pierwszym znaczącym użyciem tych technologii było mapowanie powierzchni księżyca za pomocą zdjęć z sondy kosmicznej w 1964 roku, gdzie naukowcy z Jet Propulsion Laboratory na podstawie tysięcy zdjęć odtworzyli powierzchnię księżyca. 
Przy następnej okazji mieli dostęp do 100000 zdjęć, na podstawie których mogli stworzyć mapę topograficzną, mapę kolorową oraz panoramiczną mozaikę księżyca, które przyczyniły się do pierwszego lądowania człowieka na księżycu \cite{digitalImageProcessing}.

Technologia ta jednak była ograniczona przez bardzo małe możliwości oraz trudną dostępność komputerów tamtych czasów. 
Wraz z ich rozwojem i wzrostem popularności cyfrowe przetwarzanie obrazów przestało być trudno dostępną dziedziną naukową i stopniowo zostało wdrażane do naszego codziennego życia. 
Obecnie około 80\% populacji krajów rozwiniętych jest w posiadaniu smartfona \cite{smartphones}, który na swoim wyposażeniu ma kamerę cyfrową. 
Są one ograniczone poprzez fizyczny rozmiar matrycy oraz jakość soczewek, które mogą zmieścić się w telefonach komórkowych. 
Pomimo tego dzięki przetwarzaniu obrazów można uchwycić za ich pomocą imponujące zdjęcia \cite{pixel}. 
Nowoczesne telefony są w stanie zachować na zdjęciu nocnym detale które może być ciężko zobaczyć ludzkim okiem \cite{nightMode} przez szybkie łączenie wielu obrazów o różnych długościach ekspozycji, redukcji szumów i korekcji kolorów.  

\subsection{Przegląd istniejących rozwiązań}

\subsubsection{Photoshop}
\begin{figure}[H]
    \centering
    \includegraphics{./images/Picture2.jpg}
    \caption{Adobe Photoshop}
    \label{fig:photoshop}
\end{figure}

Wydany w 1990 roku do tworzenia i edytowania obrazów rastrowych jest jednym z najbardziej popularnych programów, a jego nazwa przedarła się do języka potocznego jako zamiennik dla fotomontażu. 
Użytkownikami tego oprogramowania są przeważnie artyści, fotografowie i twórcy internetowych memów. 
To narzędzie jest przystosowane do łatwości obsługi przez mniej zaawansowanych użytkowników i wszystko posiada przyjazny graficzny interfejs użytkownika. 

Funkcje, które pozwalają na przetwarzanie obrazów często nie odnoszą się bezpośrednio do operacji zawartych w znanych bibliotekach do przetwarzania obrazów. 
Ich obsługa jest uproszczona i przystosowana do bardziej artystycznych zastosowań. 
Bez zastosowania szczególnej ostrożności - jak tworzenie nowych warstw po każdej operacji lub częste zapisywanie kopii zapasowej obrazu – przetwarzanie obrazów często jest destruktywne, nie możemy odnieść się do dowolnego momentu w ciągu naszych operacji w celu dopasowania jej parametrów. 
Przetwarzanie innego obrazu za pomocą tego samego procesu jest problematyczne, gdyż wymaga to ciągłego śledzenia warstw. Wiele operacji trzeba powtórzyć indywidualnie ponieważ nie zapisujemy jej parametrów tylko rezultat poprzedniego wykonania. Wiele bardziej zaawansowanych procesów nie jest możliwych do zrealizowania za pomocą standardowych opcji dostępnych w programie. Historia operacji jest też ograniczona i nie można wyświetlić pełnej historii na jednym ekranie - można jedynie zamieniać stan aktualny na wcześniejszy.

\subsubsection{ImageJ}
\begin{figure}[H]
    \centering
    \includegraphics[width=0.8\linewidth]{./images/Picture3.png}
    \caption{ImageJ}
    \label{fig:imagej}
\end{figure}

Wydany w 1997 roku program ImageJ został wykonany przez National Institutes of Health, został tworzony podstawowo z zamiarem analizy obrazów w zastosowaniach medycznych. 
Odtwarza dokładnie wiele operacji z bibliotek do przetwarzania obrazów i kod aplikacji jest otwartoźródłowy - każdy może go zobaczyć, pomóc w rozwoju lub stworzyć własną wersję. 
Napisany został w języku Java i ma bardzo niskie wymagania sprzętowe jak na obecne czasy.

Program jest jednak przestarzały, mogą pojawić się problemy z uruchomieniem na nowszych systemach. Interfejs użytkownika wygląda archaicznie oraz jego układ jest niepraktyczny biorąc pod uwagę przetwarzanie obrazów. 
Wszystkie operacje są schowane pod 1-2 poziomami menu. 
Operacje są destrukcyjne - zmieniają dane, które przetwarzamy więc to użytkownik musi pilnować aby mieć kopię swoich obrazów. 
Nie da się przygotować ciągów operacji wcześniej za pomocą interfejsu graficznego, trzeba użyć do tego ich języka programowania ImageJ Macro \cite{imagejbatch}. 
Od dawna nie jest rozwijany, został zastąpiony przez ImageJ2, który pozwala na przetwarzanie obrazów wielowymiarowych z dodatkowymi danymi np. z mikroskopów elektronowych, skanerów itp.

\subsubsection{Fiji}
\begin{figure}[H]
    \centering
    \includegraphics[width=0.8\linewidth]{./images/Picture4.png}
    \caption{Fiji - sukcesor ImageJ}
    \label{fig:fiji}
\end{figure}

Projekt open source oparty o ImageJ2. Oprócz podstawowych operacji wbudowanych w ImageJ posiada też wiele pluginów znacząco rozszerzających możliwości programu. 
Są one skupione na wspomaganiu przetwarzania obrazów skupionych na dziedzinie neurobiologii, ale możliwości są na tyle szerokie, że wiele innych dziedzin nauki z niego korzysta.

Podstawowe działanie programu nie różni się od ImageJ więc, wszystkie jego problemy są też tutaj obecne. Jednakże poprawiona została jego kompatybilność z najnowszymi systemami operacyjnymi.

\subsubsection{GIMP}
\begin{figure}[H]
    \centering
    \includegraphics{./images/Picture5.jpg}
    \caption{GIMP wersja 2.10}
    \label{fig:gimp}
\end{figure}

Wydany w 1998 roku GNU Image Manipulation Program jest tworzony jako darmowa konkurencja dla Photoshopa, którego licencja jest miesięczną subskrypcją. Program też w pierwszej kolejności jest tworzony z myślą o artystach, ale posiada więcej zaawansowanych opcji jak np. konwolucje macierzowe które można dowolnie edytować, dając duże możliwości filtrowania obrazów.

Pomimo odwzorowaniu większej liczby operacji ze standardowych bibliotek do przetwarzania obrazów i większej kontroli nad niektórymi z nich nadal występuje problem destruktywnego przetwarzania obrazów. Wynika to z pracy bezpośrednio na obrazie jak w Photoshopie i wymaga szczególnej uwagi przy zarządzaniu warstwami aby nie tracić bezpowrotnie swoich pośrednich operacji w dłuższym ciągu.

\subsubsection{Adobe Substance Designer}
\begin{figure}[H]
    \centering
    \includegraphics{./images/Picture6.jpg}
    \caption{Adobe Substance Designer}
    \label{fig:designer}
\end{figure}

Wydany w 2007 roku program był główną inspiracją dla tego projektu. 
Jego dużą zaletą jest to, że tworzenie algorytmów bazuje na układaniu bloczków (nodów). 
Ułożony graf jest wykonywany na obrazie, który importujemy do programu lub generujemy go od początku w nim. 
Każdy dowolny node można kliknąć i zmienić wszystkie parametry jego operacji, niezależnie w którym miejscu ciągu znajduje się.
Jego wynik i wszystkie następne operacje, które są zależne od niego są obliczane ponownie na podstawie zmienionego wyniku. 
Zmiana przetwarzanego obrazu polega na przeciągnięciu połączenia z obecnego węzła z naszym plikiem wejściowym na nowy bloczek. Następnie propagowana jest zmiana na wszystkie kolejne operacje, których wynik jest aktualizowany automatycznie.

Opisywany program ten ma wiele ograniczeń związanych z tym, że nie jest stworzony do ogólnego przetwarzania obrazów. Został natomiast stworzony do tworzenia materiałów/tekstur do grafiki komputerowej. 
Obrazy są ograniczone do boków o długości $2\textsuperscript{n}$, najlepiej kwadratowych. 
Parametry operacji są często uproszczone, ponieważ mimo większej złożoności aplikacji, jest ona nadal skierowana do artystów. 
Rodzaje operacji i wspierane formaty pikseli w pliku są przystosowane do wymagań grafiki komputerowej.

\newpage

\section{Przedstawienie użytych technologii}
\textit{W rozdziale tym opisuje się już konkretnie rozwiązania technologiczne, jakie zostały wykorzystane w pracy. Przykładowo, dla wcześniej wspomnianej pracy, w której student stworzył aplikację webową do sterowania znakami zmiennej treści, opisane będzie: dlaczego postanowił wykorzystać technologię webową; jej historię; html; css; PHP; MVC; Laravel; JavaScript; Bootstrap; VueJs. Poniżej przedstawiono, w formie podrozdziałów, przykładowe opisy w celu pokazania sugerowanej objętości.}

\subsection{Avalonia UI}
Biblioteka C\#, który posiada zestaw zdefiniowanych stylów, skryptów JS i szablonów, znacznie przyśpieszający tworzenie interfejsu graficznego stron i aplikacji internetowych. Często stosowany przez programistów, gdyż ułatwia budowanie atrakcyjnych wizualnie stron internetowych.

\subsection{Biblioteka Vuforia}
Vuforia to produkt stworzony przez amerykańskie przedsiębiorstwo Qualcomm, nabyty w 2015 roku przez  Parametric Technology Corporation [7]. Jest to zestaw narzędzi programistycznych (SDK) pozwalający na pracę z rozszerzoną rzeczywistością. Biblioteka Vuforia pozwala na wykrywanie obrazów, które wcześniej zostały przetworzone z wykorzystaniem narzędzia udostępnionego na stronie internetowej producenta. 

Alternatywą dla wykorzystanej biblioteki jest m.in. OpenCV [8]. Jest to narzędzie rozpowszechniane na licencji open source, co należy do jego zalet, jednakże jego działanie jest dużo bardziej złożone. Działanie Vuforii cechuje prostota, co więcej za jej wyborem przemawia fakt, że jest bardzo dobrze zintegrowana ze środowiskiem Unity - udostępniana jest w formie wtyczki. Jej kluczową zaletą jest to, że wykrywanie obrazu jest możliwe w bardzo szybkim tempie bez względu na kąt kamery czy też obrót obiektu. Pozwala ona także na szybkie wykrywanie oraz śledzenie kilku obiektów jednocześnie.

\subsection{Nazwa rozdziału i zakres}
\textit{Jeśli nazwy rozdziału nie da się powiązać z tematem, można go nazwać „Wykorzystane rozwiązania technologiczne” lub „Wykorzystane technologie oraz narzędzia”.}

\textit{W skład tego rozdziału również może wchodzić analiza istniejących rozwiązań, pod warunkiem, że nie ma jej w kolejnym rozdziale. Dodatkowo, opis wykorzystanego środowiska programistycznego wraz ze składowymi. Dlatego, jeśli już z góry wiecie, że lista ta może być krótka, to poszerzcie tą listę. Przykładowo:}

\begin{itemize}
    \item Git + GitHub – bo projekt będzie udostępniany w formie open-source;
    \item Task runner – Grunt / Gulp / itp.;
    \item ORM;
    \item Generator diagramów np. Class designer;
\end{itemize}
\newpage

\section{Realizacja projektu}
\textit{W rozdziale tym należy opisać proces jaki prowadził do wykonania pracy, dlaczego projekt ma taką formę. Można się podeprzeć analizą istniejących rozwiązań (jeśli nie została opisana powyżej), pracą z wykorzystaniem metodyk wytwórczych (design thinking, event storming, scenario based design, itp.). Finalnie prowadzić to ma do spisu funkcjonalności, które mogą mieć formę poniższych podrozdziałów.}

\textit{W przypadku robota edukacyjnego, student mógł skorzystać z istniejących robotów, które porównał w pracy na warsztatach RESET Young. Prowadziło to do listy plusy/minusy oraz odpowiednich wniosków. Można uznać to za wzorcowe podejście.}

\subsection{Analiza wymagań funkcjonalnych}
\textit{W tym rozdziale należy określić jakie funkcje można wykonać w aplikacji. Może mieć formę listy Aktor – Cel lub diagramu przypadków.}

\subsection{Analiza wymagań pozafunkcjonalnych}
\textit{W tym rozdziale należy określić jakie pozafunkcjonalne wymagania spełniać będzie oprogramowanie / projekt.}
\image{Overleaf_logo.png}{Zdjęcie przedstawia logo Overleaf-a}

\subsection{Proponowany model rozwiązania}
\textit{W tym rozdziale należy jaką strukturę ma aplikacja / projekt. Można do tego wykorzystać diagram DFD (lub inny zestaw diagramów UML), proponowany schemat bazy danych (choć może on również znaleźć się w opisie rozwiązania)}

\textit{W przypadku prac badawczych, należy określić, jaka będzie metodyka badań i co prowadziło do wyboru takiej metodyki.}

\subsection{Opis rozwiązania, implementacja}
\textit{W podrozdziale tym należy opisać to co jest największą samodzielną częścią projektu / aplikacji. To co student wytworzył. Można wykorzystać diagram klas, listę klas z opisem za co one odpowiadają, listę metod z opisem co robią. W przypadku bardzo zaawansowanych algorytmów, należy je przedstawić w formie blokowej lub kodu. W przypadku kodu jego formę (wygląd) przedstawiono poniżej:}

TODO

\textit{Również nazwy klas, metod itp. znajdujące się w treści powinny być pisane czcionką o stałej szerokości znaku, np. Courier New / Consolas. }

\subsection{Testowanie}
\textit{W przypadku jeśli oprogramowanie było testowane (szczególnie liczą się testy z użytkownikiem) można tutaj zaprezentować ich wyniki. Mogą też to być wyniki testów wydajnościowych. przypadku badań, może to być weryfikacja wyników (np. testy statystyczne)}
\newpage

\section{Podsumowanie}
Celem pracy było stworzenie aplikacji do przetwarzania obrazów z interfejsem graficznym.
Interakcja użytkownika z programem odbywa się przez edytor węzłowy, a parametry odpowiadają tym używanym w bibliotece OpenCV. 
Wyniki obliczeń zostają wyświetlane natychmiastowo, a parametry które można wprowadzić są sprawdzane pod kątem poprawności.  
Cel został zrealizowany.

Projekt jednak ma duży potencjał na rozwój. Inspirując się innymi projektami z podobnymi edytorami można wymienić całą listę: 
\begin{itemize}
    \item tworzenie i zapisywanie parametrów do użycia jak zmienne w programowaniu - zmiana wartości w jedynym miejscu zmienia też jej użycia w innych operacjach,
    \item możliwość łączenia kilku bloków w jeden przez użytkownika np. \textit{Blur} i \textit{EdgeDetect} ponieważ są one często używane razem,
    \item grupowanie, układanie i komentarze w edytorze,
    \item miniaturka z podglądem wyniku umieszczona na każdym węźle,
    \item eksportowanie pliku w sposób pozwalający zmianę parametrów i podgląd wyników przez zewnętrzny program.
\end{itemize}

OpenCV jest bardzo obszerną biblioteką z racji czego projekt został ograniczony do prostych operacji. Obszary do poprawy to:
\begin{itemize}
    \item zwiększenie ilości metod przetwarzania obrazów dostępnych w aplikacji,
    \item wsparcie dla innych typów danych jak wideo, chmury punktów itd.,
    \item użycie funkcji korzystających z sieci neuronowych czy uczenia maszynowego,
    \item wybór metody gdy więcej niż jedna realizuje ten sam cel.
\end{itemize}

Obecnie program działa tylko na komputerach z systemem Windows, wprowadzenie wsparcia dla uruchamiania na innych platformach pozwoli na zwiększenie potencjalnej liczby użytkowników. Do rozważenia jest skorzystanie z obliczeń w chmurze - nie każdy posiada wystarczająco szybkie urządzenia, a jako bonus użytkownicy mobilni też mogli by korzystać z aplikacji z zadowalającymi osiągami. 
\newpage
\newpage

\addcontentsline{toc}{section}{Bibliografia}
\input{sections/bibiography}
\newpage

\addcontentsline{toc}{section}{Spis rysunków}
\input{sections/imagelist}
\newpage

\addcontentsline{toc}{section}{A. Załączniki}
\section*{A. Załączniki}
\end{document}