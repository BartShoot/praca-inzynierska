Celem tej pracy jest zaprojektowanie i implementacja oprogramowania do przetwarzania obrazów opartego na bibliotece OpenCV z funkcjonalnym interfejsem graficznym.

Zakres pracy objął: analizę istniejących rozwiązań na podstawie której stwierdzono, że nie ma aktualnego i gotowego produktu który realizuje dokładnie założenie zastąpienia programistycznego podejścia do obsługi biblioteki OpenCV. Jest wiele roziązań które realizuje część biblioteki lub jej obsługa nie pozwala na przetwarzanie obrazów w niedestrukcyjny sposób bez dodatkowego wysiłku od użytkownika jak używanie warstw czy inkrementalnych zapisów. 

Jako platformę wybrano komputer osobisty z powodu znaczącej mocy obliczeniowej, dostępu do dużej ilości pamięci RAM oraz dostępu do operacji procesorów x86. \cite{x86opencv} Jako docelowy system operacyjny wybrano Microsoft Windows 11 ponieważ jest to system używany przez autora tej pracy. Do tworzenia aplikacji skorzystano z języka C\# oraz Windows Presentation Framework. 

Pomysł na ten projekt powstał dzięki doświadczeniu autora pracy w używaniu różnych systemów opartych o edytory typu węzłowego jak Adobe Substance Designer, edytor Blueprintów w Unreal Engine oraz edytor shaderów czy geometry nodes w Blenderze. Ten typ interfejsu pozwala na interakcję użytkownika z wizualnym przedstawieniem jakichś metod, funkcji czy bloków logicznych pozwalających na przesyłanie wartości w podobny sposób jak można robić to w kodzie, ale zdecydowanie bardziej przystępne dla osób które nie mają odpowiedniego doświadczenia w programowaniu.  