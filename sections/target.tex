Celem tej pracy jest zaprojektowanie i implementacja oprogramowania do przetwarzania obrazów z funkcjonalnym interfejsem graficznym.

Zakres pracy objął analizę istniejących rozwiązań, na podstawie której stwierdzono, że nie ma aktualnego i gotowego produktu, który realizuje dokładnie założenie zastąpienia programistycznego podejścia do obsługi bibliotek przetwarzania obrazów. 
Jest wiele rozwiązań, które realizuje część biblioteki lub jej obsługa nie pozwala na przetwarzanie obrazów w niedestrukcyjny sposób bez dodatkowego wysiłku od użytkownika jak używanie warstw czy inkrementalnych zapisów. 

Z powodu znaczącej mocy obliczeniowej jako platformę podstawową wybrano komputer osobisty. Ważnym atutem też jest dostęp do dużej ilości pamięci RAM oraz do operacji procesorów x86, z których biblioteki do przetwarzania obrazów czasem korzystają do akceleracji obliczeń \cite{x86opencv}. 
Jako docelowy system operacyjny wybrano Microsoft Windows 11, ponieważ jest to najnowszy system z rodziny Windows. 
Do tworzenia aplikacji skorzystano z języka C\# oraz Windows Presentation Framework. 

Pomysł na ten projekt powstał dzięki doświadczeniu autora pracy w używaniu różnych systemów opartych o edytory typu węzłowego jak Adobe Substance Designer, edytor Blueprintów w Unreal Engine oraz edytor shaderów czy geometry nodes w Blenderze. 
Ten typ interfejsu pozwala na interakcję użytkownika z wizualnym przedstawieniem jakichś metod, funkcji czy bloków logicznych pozwalających na przesyłanie wartości w podobny sposób jak można robić to w kodzie, ale zdecydowanie bardziej przystępne dla osób, które nie mają odpowiedniego doświadczenia w programowaniu.

Dużą zaletą tego podejścia w porównaniu do edycji w programach posługujących się warstwami jest to, że możemy wrócić w każdym momencie do dowolnej poprzedniej operacji. 
Użytkownik może zmienić wartość operacji, zmienić ich kolejność i w żadnym momencie nie traci wyniku operacji, ponieważ każda zmiana powoduje obliczenie rezultatu jeszcze raz. 
Przeliczenie całego drzewa na nowym obrazie wejściowym i otrzymanie wyniku, którego się spodziewamy to zmiana jednego parametru.