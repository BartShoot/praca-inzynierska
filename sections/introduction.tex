
Przetwarzanie obrazów to dziedzina informatyki zajmująca się automatyzacją zadań związanych z obrazami. Obejmuje ona szeroki zakres zagadnień, takich jak sztuczna inteligencja, uczenie maszynowe, analiza danych i grafika komputerowa.

Oprogramowanie do przetwarzania obrazów jest wykorzystywane w wielu różnych dziedzinach, takich jak medycyna, przemysł, bezpieczeństwo i rozrywka. W medycynie jest wykorzystywane do diagnostyki, obrazowania medycznego i chirurgii. W przemyśle jest wykorzystywane do kontroli jakości, automatyzacji i inspekcji. W bezpieczeństwie jest wykorzystywane do monitorowania, rozpoznawania twarzy i identyfikacji. W rozrywce jest wykorzystywane do tworzenia efektów specjalnych, animacji i gier wideo.

Jedną z popularnych bibliotek do przetwarzania obrazów jest OpenCV. OpenCV to darmowy, otwartoźródłowy pakiet oprogramowania dla systemów operacyjnych Windows, macOS i Linux. Zapewnia szeroki zakres funkcji do przetwarzania obrazów, w tym: podstawowe operacje takie jak filtrowanie, skalowanie i konwersja formatów, wykrywania i rozpoznawania obiektów czy obsługę algorytmów do analizy wideo.

W ramach czasu, który został poświecony na niniejszy projekt zakres możliwości ograniczono do prostych operacji przetwarzania obrazów bez uwzględniania wideo oraz wszelkiego rodzaju danych oprócz samych obrazów. 

OpenCV jest często wykorzystywany do tworzenia oprogramowania do przetwarzania obrazów. Jednak jego interfejs użytkownika jest dostosowany do programistów, a nie do użytkowników końcowych. Oznacza to, że użytkownicy końcowi muszą mieć podstawowe umiejętności programowania, aby korzystać z oprogramowania stworzonego w oparciu o OpenCV.

W tej pracy naukowej proponujemy nowe oprogramowanie do przetwarzania obrazów oparte na bibliotece OpenCV. Oprogramowanie to posiada funkcjonalny interfejs graficzny, który ułatwia użytkownikom końcowym korzystanie z niego.