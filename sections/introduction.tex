\textit{W tym miejscu należy umieścić wstęp opisujący dlaczego ten temat jest istotny oraz motywację studenta w jego wyborze. Poniżej przykład jak zacząć nawiązanie do dziedziny, tak aby nie zaczynać pracy od oklepanej frazy „W dzisiejszych czasach …”:}

Nauka programowania od zawsze uznawana była za bardzo trudne zadanie dla nowicjuszy. Podejmowane przez nich próby pisania kodu często kończyły się niepowodzeniem i szybką rezygnacją.

Techniki biometryczne pozwalają na ochronę zasobów na urządzeniach bez potrzeby wprowadzania pinów, haseł oraz skomplikowanych wzorów, które mogą zostać zapomniane lub przechwycone przez osoby postronne. Wystarczy przyłożyć palec, a urządzenie zostaje odblokowane w ułamku sekundy. Jest to rozwiązanie bardzo wygodne dla użytkowników.

Efektywna komunikacja jest czynnikiem koniecznym do zagwarantowania rozwoju, stabilizacji, jak i również bezpieczeństwa niezależnie od płaszczyzny, na której zachodzi. Aby można było mówić o sukcesie procesu informowania, należy szczegółowo przeanalizować domenę w jakiej ma ona zachodzi oraz wybrać właściwą metodę przesyłania danych.

\textit{Jak widać, w każdym przypadku akapit zaczął się od określenia jakiejś pozytywnej lub negatywnej strony związanej z tematem pracy. Pozwala to na łatwe rozwinięcie treści.}

\textit{Jeśli brakuje wam weny, to wstępie możecie poruszyć opis problemów związanych z tematem, zastosowanie danej technologii, badania rynkowe np. o wzrastającej popularności danej technologii.}

\textit{Poniżej przedstawiłem przykładowe zakończenia z motywacją:}

Na rynku dostępnych jest wiele modeli robotów edukacyjnych, jednak większość 
z nich jest bardzo droga lub trudno dostępna dla szkół. Istnieje więc potrzeba stworzenia optymalnego modelu, łatwego w obsłudze, a zarazem będącego w przystępnej cenie. 

Praca ta ma na celu przedstawienie i porównanie różnych technik analizy interfejsu użytkownika tak, aby to, co wydaje się być intuicyjne i proste dla konstruującego stronę programisty, faktycznie takim się stało dla odbiorcy – użytkownika owego produktu.

\textbf{Całość, razem z zakończeniem powinna mieścić się na jednej stronie.}