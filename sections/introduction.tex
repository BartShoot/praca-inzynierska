
Przetwarzanie obrazów to zbiór wielu algorytmów które należy rozróżnić na dwie dziedziny ze względu na rodzaj reprezentowanych i manipulowanych informacji.
Pierwsza dotyczy analogowej reprezentacji danych \cite{signalProcessing}, gdzie informacje są w formie ciągłych sygnałów odwzorowujących obrazy. 
Operacje na nich są realizowane przez modyfikacja impulsów elektrycznych reprezentujących informacje. 
Technologia ta umożliwiła rozwój transmisji telewizyjnej \cite{times1926}.
Cyfrowe przetwarzanie obrazów \cite{digitalImageProcessing} jest nowszą technologią.
Dane są reprezentowane poprzez wielowymiarowe macierze wartości \cite{OpenCVMat}. 
Mogą one odpowiadać kolorowi obrazu w danym miejscu, wartościom chmury punktów lub innym informacjom które mogą zostać przetworzone cyfrowo. 
Modyfikacja odbywa się przez zmiany wartości macierzy. 
Obie dziedziny mają swoje miejsca i zastosowania ale niniejsza praca zajmie się cyfrowym przetwarzaniem obrazów.

Przetwarzania obrazów jest wykorzystywane w wielu różnych dziedzinach: 
\begin{itemize}
    \item \textbf{Medycyna:} Diagnostyka, planowanie zabiegów chirurgicznych i monitorowanie stanu pacjentów \cite{skszymon2023}. 
    Dane medyczne uzyskiwane w formie obrazów cyfrowych są ustandaryzowane przez Digital Imaging and Communications in Medicine \cite{DICOMPart01}.
    \item \textbf{Przemysł:} Kontrola jakości, automatyzacja i inspekcja. Systemy wizyjne potrafią szybciej i dokładniej odkryć defekty produktów na liniach produkcyjnych \cite{Nix2024}.
    \item \textbf{Bezpieczeństwo:} Monitorowanie i rozpoznawanie twarzy. Drony ze wsparciem przetwarzania obrazów są używane przez służby bezpieczeństwa w trudnych warunkach \cite{dji}.
    \item \textbf{Rozrywka:} Tworzenie efektów specjalnych dla filmów, gier i animacji. Transformacje przestrzeni kolorów i korekcje nagrań w celu poprawienia efektu końcowego \cite{cinema}.
\end{itemize}

Przetwarzanie obrazów jako manipulacja danymi cyfrowymi została stworzona przez programistów \cite{computerProcessing}. 
Kod to podstawowa wersja interakcji człowieka z komputerem, ale większość użytkowników tych urządzeń nie potrafi programować. 
Żeby dziedzina tej nauki była przystępna dla większej liczby osób należy stworzyć aplikację z prostym i funkcjonalnym graficznym interfejsem użytkownika. 

W tej pracy naukowej proponujemy nowe oprogramowanie do przetwarzania obrazów oparte na bibliotece OpenCV. 
Aplikacja posiada funkcjonalny interfejs graficzny, który ułatwia użytkownikom końcowym obsługę tej biblioteki bez potrzeby znajomości programowania.