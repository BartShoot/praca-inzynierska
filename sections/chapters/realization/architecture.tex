
Projekt ten powstał w technologii Windows Presentation Foundation. 
Jedną z metod rozdzielenia widoku od logiki biznesowej w interfejsie tworzonym w WPF jest wzorzec MVVM (ang. \textit{Model-View-ViewModel}). 
Jego cel to jasny podział aplikacji na \textit{Model} - dane naszej aplikacji, ich interakcja ze sobą i implementacja mechanizmów działania oprogramowania. 
\textit{View} odpowiada tylko za wyświetlanie danych użytkownikowi oraz przyjmowanie jego interakcji jak kliknięcia czy wprowadzanie informacji. 
Za to klasy \textit{ViewModel} zajmują sie na połączeniu modelu i widoku konwertując obiekty z części biznesowej na dane które mają dotrzeć do użytkownika. 
Ich zadaniem jest też obsługa danych wprowadzanych przez interfejs aplikacji.

\subsubsection{Model}

\begin{figure}[H]
    \centering
    \includegraphics[width=1\linewidth]{images/Picture11.png}
    \caption{Diagram operacji. Opracowanie własne.}
    \label{fig:modelDiag}
\end{figure}

W NoodleCV warstwa \textit{Model} jest minimalna \autoref{fig:modelDiag}. 
Posiada interfejs IOperation odpowiadający za formę wszystkich implementacji operacji. 
Posiadają generyczna kolekcje wejść \textit{Inputs} oraz wyjść \textit{Outputs}.
Jedyna metoda zdefiniowana w nich to wykonanie operacji - zwraca ona obiekt \textit{Result}.
Informuje ona następne warstwy czy operacja się powiodła, jeżeli nie to dlaczego. 
Listę błędów otrzymujemy dzięki sprawdzeniu wejść za pomocą biblioteki Fluent Validation. 
Każda operacja wymaga swojego własnego walidatora danych i zapewnia tym bezpieczne działanie aplikacji - użytkownik nie powinien być w stanie doprowadzić programu do błędu złym parametrem operacji.

\begin{figure}[H]
    \centering
    \includegraphics[width=0.6\linewidth]{images/Picture12.png}
    \caption{Diagram generycznych typów przechowujących dane operacji. Opracowanie własne.}
    \label{fig:input}
\end{figure}

Wejścia i wyjścia parametrów operacji mogą być w wielu różnych typach więc kolekcje przyjmujące te dane musiały być generyczne. 
Daje to dużo większą elastyczność ale może utrudnić implementację. 
Tworząc operacje i ich późniejsze \textit{ViewModel}'e trzeba zachować szczególną ostrożność. 
Przy tworzeniu nowych modeli zakładamy że obiekt w tym miejscu kolekcji będzie miał odpowiedni typ. 
Przy odczytywaniu danych z tej kolekcji nie ma informacji co to za klasa, należy poprawnie narzucić jej rodzaj inaczej napotkamy błąd i aplikacja wyłącza się.
Pomimo tych problemów warto korzystać z tej struktury ponieważ niektóre operacje przyjmują tylko ścieżkę do pliku w formie tekstu, inne mogą nie mieć żadnego wyjścia ale wszystkie pasują do jednego wspólnego interfejsu.

\subsubsection{View} 

\begin{figure}[H]
    \centering
    \includegraphics[width=0.8\linewidth]{images/Picture19.png}
    \caption{Diagram klas używanych do tworzenia interfejsu. Opracowanie własne.}
    \label{fig:viewArch}
\end{figure}

Warstwa widoku składa się z głównego okna, edytora i jego elementów.
Dodatkowo użyto klasy statycznej posiadającej metody pozwalające na konwersję obrazów z formatu używanego przez operacje na format akceptowany przez wyświetlanie obrazu w WPF.
Drugi konwerter zajmuje się pobieraniem danych z obrazu przetwarzanego przez operację i stworzenie opisu zawierający jego rozmiar oraz ilość kanałów koloru.

\begin{figure}[H]
    \centering
    \includegraphics[width=1\linewidth]{images/Picture13.png}
    \caption{Główne okno aplikacji z przykładowymi operacjami. Opracowanie własne.}
    \label{fig:window}
\end{figure}

Aplikacja posiada jedno okno \autoref{fig:window}. 
Użyta została klasa \textit{UiWindow} z biblioteki WPF UI \cite{wpfui} pozwalająca łatwo zmodyfikować górny pasek by pasował do nowoczesnych systemów operacyjnych. 

\begin{figure}[H]
    \centering
    \includegraphics[width=0.6\linewidth]{images/Picture14.png}
    \caption{Menu aplikacji. Opracowanie własne.}
    \label{fig:mainmenu}
\end{figure}

Menu pozwala na zapisywanie odczytywanie lub czyszczenie stanu aplikacji. 
Opcje te są też przypisane znanym skrótom klawiszowym w celu ułatwienia ich używania.

Zawartość okna znajduje się w osobnym widoku edytora. Jest on podzielony na 3 główne elementy. Edytor (\autoref{fig:editor}) gdzie dodaje się nowe operacje, podgląd (\autoref{fig:preview}) gdzie pojawiają się wyniki operacji i panel od modyfikacji parametru (\autoref{fig:params}).

\begin{figure}[H]
    \centering
    \includegraphics[width=1\linewidth]{images/Picture15.png}
    \caption{Widok edytora. Opracowanie własne.}
    \label{fig:editor}
\end{figure}

W tej części widoku użytkownik może dodawać nowe operacje, łączyć je, przeciągać oraz wybierać i zmieniać parametry wybranych bloczków. 
Operacje dodajemy za pomocą menu kontekstowego (\autoref{fig:context}) umieszczonego w edytorze. 
Otwiera się je za pomocą kliknięcia prawego przycisku myszy na jego obszarze.
Węzły stworzone w ten sposób posiadają nagłówek odpowiadający nazwie operacji którą reprezentują oraz połączenia wejściowe i wyjściowe reprezentujące możliwe połączenia.
Po uzyskaniu rezultatu pojawia się też informacja jakiego rozmiaru jest obraz znajdujący się na wyjściu i ile kanałów posiada - czy to obraz biało-czarny lub kolorowy.
Są to elementy biblioteki Nodify \cite{nodify} ze zmodyfikowanymi stylami w celu utrzymania jednolitej stylistyki projektu.

\begin{figure}[H]
    \centering
    \includegraphics[width=0.6\linewidth]{images/Picture18.png}
    \caption{Menu kontekstowe edytora. Opracowanie własne.}
    \label{fig:context}
\end{figure}


Po dodaniu operacji zostaje ona automatycznie przypisana do podglądu obrazu (\autoref{fig:preview}) umieszczonego pod edytorem. Użytkownik może później zmienić i wybrać które elementy są wyświetlane za pomocą klawiszy `1' i `2'.

\begin{figure}[H]
    \centering
    \includegraphics[width=1\linewidth]{images/Picture16.png}
    \caption{Podgląd wyników operacji. Opracowanie własne.}
    \label{fig:preview}
\end{figure}

\begin{figure}[H]
    \centering
    \includegraphics[width=0.6\linewidth]{images/Picture17.png}
    \caption{Edycja parametrów. Opracowanie własne.}
    \label{fig:params}
\end{figure}

Ostatni element widoku aplikacji to panel (\autoref{fig:params}) gdzie można edytować parametry wybranej operacji. 
Każdy węzeł potrzebuje swojego osobnego widoku który poprawnie odwzorowuje parametry potrzebne do wykonania operacji. 
Są one przypisane do pól w odpowiednich ViewModeli i każda zmiana użytkownika jest przesyłana do niego w celu zaktualizowania wyników operacji.

\begin{figure}[H]
    \centering
    \includegraphics[width=0.8\linewidth]{images/Picture31.png}
    \caption{Komunikat z błędem. Opracowanie własne.}
    \label{fig:error}
\end{figure}

Po wprowadzeniu nieprawidłowej wartości pojawia się czerwony komunikat (\autoref{fig:error}) wskazujący dlaczego operacja nie mogła się wykonać.

\subsubsection{ViewModel}

W przypadku tego projektu warstwa komunikacji między modelem i widokiem jest najbardziej obszerna. 
Większość z tych klas dziedziczy po ViewModelBase wziętym z \textit{BindableBase} \cite{prismlibraryprism} z biblioteki \textit{Prism} \cite{prismlibrary}. 
Znajdują się w nim metody pomagające z obsługą interfejsu IPropertyChanged. 
Jest on kluczowy w tworzeniu aplikacji reagującej natychmiastowo na dane wejściowe od użytkownika w technologii WPF. 
Dają one znać komponentom do których przypisane są wartości, że się zmieniły i trzeba je zaktualizować.


\begin{figure}[H]
    \centering
    \includegraphics[width=1\linewidth]{images/Picture20.png}
    \caption{Diagram ViewModeli operacji. Opracowanie własne.}
    \label{fig:vmDiagOperation}
\end{figure}

Każda operacja potrzebuje swój ViewModel. Dbają one o poprawne wyświetlanie i przypisanie wartości z modelu. Zajmują się też uruchomieniem obliczeń i zwróceniem ich rezultatu. Każdy z nich posiada też metodę pozwalającą na skopiowanie parametrów z innej operacji. Jest to wykorzystane przy odczytywaniu danych z zapisanego stanu.

\begin{figure}[H]
    \centering
    \includegraphics[width=1\linewidth]{images/Picture21.png}
    \caption{Diagram ViewModeli aplikacji. Opracowanie własne.}
    \label{fig:vmDiagEditor}
\end{figure}

Pozostałe ViewModele dalej korzystają z tej samej klasy bazowej.
MainViewModel tworzy jedynie komendy do których odnosi się menu główne, większość logiki programu jest umieszczona w EditorViewModelu.
Jest tam umieszczona kolekcja przechowująca stworzone węzły i ich stany oraz połączenia między nimi.
Klasa NodeViewModel odpowiada bloczkom w edytorze. Przechowują one ViewModel odpowiedniej operacji, ale też pozycje na ekranie wraz z wejściami i wyjściami. 
Są to obiekty typu ConnectorViewModel między którymi tworzą się połączenia przesyłające dane między węzłami. 
