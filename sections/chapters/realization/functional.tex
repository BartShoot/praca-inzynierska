Wykazano listę funkcjonalności, które należy zaimplementować, aby móc uznać aplikację za spełniającą swoje zadanie. 

\begin{itemize}
    \item \textbf{Przetwarzanie obrazów:} Pierwszym i głównym wymaganiem jest możliwość przetwarzania obrazów. Program musi być w stanie wczytać obraz, wykonać na nim operacje i zapisać ich wynik.
    \item \textbf{Tworzenie ciągów operacji:} Wykonanie pojedynczej funkcji OpenCV nie jest skomplikowanym zadaniem i robią to już inne programy, a nawet aplikacje mobilne dostępne od razu w przeglądarce. Aby aplikacja spełniała swoje zadanie jako obsługę biblioteki programistycznej bez użycia kodu potrzebna jest funkcjonalność łączenia wielu operacji wykonywanych jedna po drugiej automatyczne.
    \item \textbf{Zapisywanie pracy:} Użytkownik ma posiadać opcję zapisania do pliku ciągu operacji, które stworzył. Biorąc inspirację z różnego typu narzędzi dostępnych na rynku \cite{designer,photoshop} możliwość powrotu do tworzonego wcześniej pliku jest kluczowa. Bez zapisywania czy otwierania poprzedniego stanu programu praca włożona w efekt końcowy przepada wraz z zamknięciem aplikacji.
\end{itemize}