Definiują one jakie doświadczenie powinien mieć użytkownik w trakcie korzystania z oprogramowania do przetwarzania obrazów.
\begin{itemize}
    \item \textbf{Proste w obsłudze:} Aplikacja powina być zbudowana w taki sposób, by osoba potrafiąca obsługiwać komputer mogła z minimalną ilością lub brakiem instrukcji zacząć korzystać z programu.
    \item \textbf{Natychmiastowe wyniki:} Przetwarzanie obrazów powinno odbywać się na tyle szybko by użytkownik widział zmiany które wprowadza na bieżąco. Otrzymanie rezultatu ma odbywać się od razu po wprowadzeniu nowych parametrów operacji.
    \item \textbf{Zastąpienie programowania do przetwarzania obrazów:} Program powinien być na tyle użyteczny żeby korzystać z niego zamiast obsługiwać OpenCV za pomocą kodu. 
\end{itemize}