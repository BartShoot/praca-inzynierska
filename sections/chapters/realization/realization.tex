
Przed rozpoczęciem pracy nad aplikacją stwierdzono możliwe zastosowania, które skorzystałyby z interfejsu graficznego do biblioteki przetwarzania obrazów.
Na ich podstawie zostały zdefiniowane wymagania funkcjonalne - dotyczące możliwości programu oraz wymagania niefunkcjonalne - mówiące o tym jak użytkownik będzie korzystał z projektu.

\subsection{Zastosowania}

\begin{itemize}
    \item \textbf{Tworzenie gotowych ciągów operacji:} 
    Istnieje potrzeba przetwarzania małej liczby obrazów lub problem jest mało skomplikowany. 
    Problematyczne jest wtedy szukanie programisty, który stworzy dla nas program, ale bez wiedzy o programowaniu zrobienie tego w kodzie samodzielnie może być zbyt ciężkie. 
    Program z interfejsem graficznym pozwoli na stworzenie procesów nawet przez osoby mniej techniczne.
    \item \textbf{Uczenie się:} 
    Osoby chcące poznać techniki przetwarzania obrazów będą mogły w prosty sposób bez znajomości programowania zobaczyć na własne oczy działanie funkcji, ich interakcję ze sobą oraz łatwo dopasowywać ich parametry. 
    \item \textbf{Komunikacja:} 
    Dzięki temu oprogramowaniu programista może łatwiej i efektywniej porozumieć się z osobami mniej zaznajomionymi z programowaniem i dużo szybciej wprowadzać zmiany w porównaniu z pisaniem kodu i jego kompilacją przy każdej zmianie.
\end{itemize}

\subsection{Analiza wymagań projektu}
Oprogramowanie opisane w tej pracy zostało nazwane \textbf{NoodleCV}. 
Następne podrozdziały przedstawią wymagania z jakimi trzeba było się zmierzyć w trakcie tworzenia aplikacji.

\subsubsection{Wymagania funkcjonalne}
Wykazano listę funkcjonalności, które należy zaimplementować, aby móc uznać aplikację za spełniającą swoje zadanie. 

\begin{itemize}
    \item \textbf{Przetwarzanie obrazów:} Pierwszym i głównym wymaganiem jest możliwość przetwarzania obrazów. Program musi być w stanie wczytać obraz, wykonać na nim operacje i zapisać ich wynik.
    \item \textbf{Tworzenie ciągów operacji:} Wykonanie pojedynczej funkcji OpenCV nie jest skomplikowanym zadaniem i robią to już inne programy, a nawet aplikacje mobilne dostępne od razu w przeglądarce. Aby aplikacja spełniała swoje zadanie jako obsługę biblioteki programistycznej bez użycia kodu potrzebna jest funkcjonalność łączenia wielu operacji wykonywanych jedna po drugiej automatyczne.
    \item \textbf{Zapisywanie pracy:} Użytkownik ma posiadać opcję zapisania do pliku ciągu operacji, które stworzył. Biorąc inspirację z różnego typu narzędzi dostępnych na rynku \cite{designer,photoshop} możliwość powrotu do tworzonego wcześniej pliku jest kluczowa. Bez zapisywania czy otwierania poprzedniego stanu programu praca włożona w efekt końcowy przepada wraz z zamknięciem aplikacji.
\end{itemize}

\subsubsection{Wymagania niefunkcjonalne}
Niefunkcjonalne wymagania?
\begin{itemize}
    \item Łatwość użytkowania
\end{itemize}

\subsection{Architektura aplikacji}


Projekt ten powstał w technologii Windows Presentation Foundation. 
Jedną z metod rozdzielenia widoku od logiki biznesowej jest wzorzec MVVM (ang. \textit{Model-View-ViewModel}). Jej celem jest

\subsection{Zaimplementowane operacje}

Dla każdej operacji zostanie wczytany ten sam obraz i zamieszczone poniżej zrzuty ekranu będą prezentowały podgląd zdjęcia przed i po, a także panel z parametrami.
Zaimplementowano najpopularniejsze \cite{mostpop} metody, które pasowały do obecnych założeń projektu - operacja powinna przetwarzać obraz i zwracać obraz jako wynik.
\subsubsection{Blur}
\begin{figure}[H]
    \centering
    \includegraphics[width=1\linewidth]{images/Picture22.png}
    \caption{Operacja \textit{Blur}. Opracowanie własne.}
    \label{fig:blur}
\end{figure}

Zaimplementowana w oparciu o \textit{Gaussian Blur} \cite{gauss}. Parametr \textit{Size} musi być nieparzysty ponieważ opisuje on rozmiar maski na podstawie której wartość piksela jest uśredniana - potrzeby jest środek. \textit{Strength} opisuje natomiast odchylenie standardowe jakie będzie zastosowane przy przeprowadzaniu operacji.

\subsubsection{ChangeColorspace}

\begin{figure}[H]
    \centering
    \includegraphics[width=1\linewidth]{images/Picture23.png}
    \caption{Operacja \textit{ChangeColorspace}. Opracowanie własne.}
    \label{fig:colorconv}
\end{figure}

Jest to funkcja \textit{CvtColor} \cite{cvtcol}. Konwertuje ona przestrzeń kolorów obrazu i dopasowuje ilość kanałów.
Jej jedynym parametrem jest kod z rodzajem konwersji np. \textit{BGR2GRAY} oznaczający przejście ze zwykłego kolorowego zdjęcia na biało-czarne.

\subsubsection{Crop}

\begin{figure}[H]
    \centering
    \includegraphics[width=1\linewidth]{images/Picture24.png}
    \caption{Operacja \textit{Crop}. Opracowanie własne.}
    \label{fig:crop}
\end{figure}

Tutaj implementacja to nie metoda ale stworzenie nowego \textit{Mat}'a \cite{mat}.
Do jego konstruktora zostają dodane koordynaty startowe oraz rozmiar nowego prostokąta który wyznacza co będzie znajdowało się w nowym obrazie.

\subsubsection{EdgeDetect}

\begin{figure}[H]
    \centering
    \includegraphics[width=1\linewidth]{images/Picture25.png}
    \caption{Operacja \textit{EdgeDetect}. Opracowanie własne.}
    \label{fig:edgeDetect}
\end{figure}  

Metoda \textit{Canny} \cite{canny} to jeden z kilku dostępnych algorytmów wykrywania krawędzi. Jako parametry wejściowe należy podać niższy i wyższy próg akceptacji.
Jeżeli krawędź jest mało wyraźna zostaje odrzucona przez algorytm. 
Wartości powyżej drugiego parametru są uznawane za krawędź, a te które są pomiędzy zostają włączone do wyniku jako element łączący krawędzie jeżeli taki jest. 

\subsubsection{Resize}

\begin{figure}[H]
    \centering
    \includegraphics[width=1\linewidth]{images/Picture26.png}
    \caption{Operacja \textit{Resize}. Opracowanie własne.}
    \label{fig:resize}
\end{figure} 

\textit{Resize} jest wykonywane przez metodę o tej samej nazwie z biblioteki OpenCV \cite{resize}. Po podaniu nowych wymiarów otrzymujemy obraz stworzony na podstawie oryginalnego ze zmienioną rozdzielczością.

\subsection{Przykładowe ciągi operacji}

\begin{figure}[H]
    \centering
    \includegraphics[width=1\linewidth]{images/Picture27.jpg}
    \caption{Wykrywanie krawędzi na wybranym fragmencie obrazu. Opracowanie własne.}
    \label{fig:socket}
\end{figure} 

\begin{figure}[H]
    \centering
    \includegraphics[width=1\linewidth]{images/Picture28.jpg}
    \caption{Wykrywanie krawędzi dla uwydatnienia liter. Opracowanie własne.}
    \label{fig:keyboard}
\end{figure} 

\begin{figure}[H]
    \centering
    \includegraphics[width=1\linewidth]{images/Picture29.jpg}
    \caption{Obcięcie zdjęcia i zmiana na odcienie szarości. Opracowanie własne.}
    \label{fig:person}
\end{figure} 

\begin{figure}[H]
    \centering
    \includegraphics[width=1\linewidth]{images/Picture30.jpg}
    \caption{Abstrakcyjne edytowanie. Opracowanie własne.}
    \label{fig:abstract}
\end{figure} 