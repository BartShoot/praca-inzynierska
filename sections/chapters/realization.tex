\textit{W rozdziale tym należy opisać proces jaki prowadził do wykonania pracy, dlaczego projekt ma taką formę. Można się podeprzeć analizą istniejących rozwiązań (jeśli nie została opisana powyżej), pracą z wykorzystaniem metodyk wytwórczych (design thinking, event storming, scenario based design, itp.). Finalnie prowadzić to ma do spisu funkcjonalności, które mogą mieć formę poniższych podrozdziałów.\cite{test}}

\textit{W przypadku robota edukacyjnego, student mógł skorzystać z istniejących robotów, które \cite{texbook} porównał w pracy na warsztatach RESET Young. Prowadziło to do listy plusy/minusy oraz odpowiednich wniosków. Można uznać to za wzorcowe podejście.}

\subsection{Analiza wymagań funkcjonalnych}
\textit{W tym rozdziale należy określić jakie funkcje można wykonać w aplikacji. Może mieć formę listy Aktor – Cel lub diagramu przypadków.}

\subsection{Analiza wymagań pozafunkcjonalnych}
\textit{W tym rozdziale należy określić jakie pozafunkcjonalne wymagania spełniać będzie oprogramowanie / projekt.}
\image{Overleaf_logo.png}{Zdjęcie przedstawia logo Overleaf-a}

\subsection{Proponowany model rozwiązania}
\textit{W tym rozdziale należy jaką strukturę ma aplikacja / projekt. Można do tego wykorzystać diagram DFD (lub inny zestaw diagramów UML), proponowany schemat bazy danych (choć może on również znaleźć się w opisie rozwiązania)}

\textit{W przypadku prac badawczych, należy określić, jaka będzie metodyka badań i co prowadziło do wyboru takiej metodyki.}

\subsection{Opis rozwiązania, implementacja}
\textit{W podrozdziale tym należy opisać to co jest największą samodzielną częścią projektu / aplikacji. To co student wytworzył. Można wykorzystać diagram klas, listę klas z opisem za co one odpowiadają, listę metod z opisem co robią. W przypadku bardzo zaawansowanych algorytmów, należy je przedstawić w formie blokowej lub kodu. W przypadku kodu jego formę (wygląd) przedstawiono poniżej:}

TODO

\textit{Również nazwy klas, metod itp. znajdujące się w treści powinny być pisane czcionką o stałej szerokości znaku, np. Courier New / Consolas. }

\subsection{Testowanie}
\textit{W przypadku jeśli oprogramowanie było testowane (szczególnie liczą się testy z użytkownikiem) można tutaj zaprezentować ich wyniki. Mogą też to być wyniki testów wydajnościowych. przypadku badań, może to być weryfikacja wyników (np. testy statystyczne)}
