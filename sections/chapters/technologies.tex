\textit{W rozdziale tym opisuje się już konkretnie rozwiązania technologiczne, jakie zostały wykorzystane w pracy. Przykładowo, dla wcześniej wspomnianej pracy, w której student stworzył aplikację webową do sterowania znakami zmiennej treści, opisane będzie: dlaczego postanowił wykorzystać technologię webową; jej historię; html; css; PHP; MVC; Laravel; JavaScript; Bootstrap; VueJs. Poniżej przedstawiono, w formie podrozdziałów, przykładowe opisy w celu pokazania sugerowanej objętości.}

\subsection{Avalonia UI}
Biblioteka C\#, który posiada zestaw zdefiniowanych stylów, skryptów JS i szablonów, znacznie przyśpieszający tworzenie interfejsu graficznego stron i aplikacji internetowych. Często stosowany przez programistów, gdyż ułatwia budowanie atrakcyjnych wizualnie stron internetowych.

\subsection{Biblioteka Vuforia}
Vuforia to produkt stworzony przez amerykańskie przedsiębiorstwo Qualcomm, nabyty w 2015 roku przez  Parametric Technology Corporation [7]. Jest to zestaw narzędzi programistycznych (SDK) pozwalający na pracę z rozszerzoną rzeczywistością. Biblioteka Vuforia pozwala na wykrywanie obrazów, które wcześniej zostały przetworzone z wykorzystaniem narzędzia udostępnionego na stronie internetowej producenta. 

Alternatywą dla wykorzystanej biblioteki jest m.in. OpenCV [8]. Jest to narzędzie rozpowszechniane na licencji open source, co należy do jego zalet, jednakże jego działanie jest dużo bardziej złożone. Działanie Vuforii cechuje prostota, co więcej za jej wyborem przemawia fakt, że jest bardzo dobrze zintegrowana ze środowiskiem Unity - udostępniana jest w formie wtyczki. Jej kluczową zaletą jest to, że wykrywanie obrazu jest możliwe w bardzo szybkim tempie bez względu na kąt kamery czy też obrót obiektu. Pozwala ona także na szybkie wykrywanie oraz śledzenie kilku obiektów jednocześnie.

\subsection{Nazwa rozdziału i zakres}
\textit{Jeśli nazwy rozdziału nie da się powiązać z tematem, można go nazwać „Wykorzystane rozwiązania technologiczne” lub „Wykorzystane technologie oraz narzędzia”.}

\textit{W skład tego rozdziału również może wchodzić analiza istniejących rozwiązań, pod warunkiem, że nie ma jej w kolejnym rozdziale. Dodatkowo, opis wykorzystanego środowiska programistycznego wraz ze składowymi. Dlatego, jeśli już z góry wiecie, że lista ta może być krótka, to poszerzcie tą listę. Przykładowo:}

\begin{itemize}
    \item Git + GitHub – bo projekt będzie udostępniany w formie open-source;
    \item Task runner – Grunt / Gulp / itp.;
    \item ORM;
    \item Generator diagramów np. Class designer;
\end{itemize}